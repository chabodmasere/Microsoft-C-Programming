\documentclass{article}
\usepackage{graphicx} % Required for inserting images
\usepackage{listings}
\usepackage{xcolor}

\definecolor{codebg}{rgb}{0.97,0.97,0.97}
\definecolor{codekw}{rgb}{0.15,0.15,0.60}
\definecolor{codestr}{rgb}{0.60,0.10,0.10}
\definecolor{codecom}{rgb}{0.10,0.50,0.10}
\definecolor{codenm}{rgb}{0.35,0.35,0.35}

\lstdefinestyle{vscode}{
  backgroundcolor=\color{codebg},
  basicstyle=\ttfamily\small,
  numbers=left,
  numberstyle=\tiny\color{codenm},
  frame=single,
  rulecolor=\color{codenm},
  breaklines=true,
  tabsize=2,
  showstringspaces=false,
  keywordstyle=\bfseries\color{codekw},
  commentstyle=\itshape\color{codecom},
  stringstyle=\color{codestr}
}

\lstset{style=vscode}


\title{C++ Programming Fundamentals}
\author{Chabod Masere}
\date{February 2026}

\begin{document}
\begin{figure}
    \centering
    \includegraphics[width=0.35\textwidth]{images/microsoft.png}
    \label{fig:myimage}
\end{figure}
\maketitle

\section{Module}
\subsection{Introduction to C++ Syntax and Structure}
\subsubsection{The Anatomy of "Hello, World!"}

The "Hello, World!" is synonymous in the programming world and is most often the first step for any new programmer, including myself. Despite its simplicity, it ultimately lays the groundwork for understanding the fundamental structure of C++ programs. The program's task is relatively singular: it outputs "Hello, World!" to the console. This is the surface of the building blocks of C++ projects.



Here is the snippet of the code: 

\lstinputlisting[language=C++]{helloworld.cpp}






















\newpage
\subsection{Data Types and Variables in C++}
\subsection{Operators and Control Flow}
\subsection{Code Creation and Compilation}
\subsection{Hands-On Course Project}

\end{document}
